\subsection{Bot Detection}

To prevent automated programs, or bots, from abusing online services, the widely adopted solution is to deploy a CAPTCHA system. The early form of CAPTCHA typically requires users to identify text from a distorted image. For Google reCAPTCHA, the most popular CAPTCHA service, user inputs are also collected to help Google digitalise printed documents~\cite{von2008recaptcha}. However, text-based CAPTCHA schemes have been proven to be insecure as machines achieved 99.8\% success rate in identifying distorted text~\cite{yan2008low,5395072,goodfellow2013multi}. Audio-based CAPTCHAs have also been used to assist visually impaired people, but they are difficult to solve, with over half of users failed during their first attempt~\cite{tasidou2012user}. Therefore, CAPTCHA service providers, such as Google, started to test image-based CAPTCHA schemes, which require users to select images that match given description~\cite{Google2014}. In the same time, reCAPTCHA users also help Google label images for free. Nevertheless, Sivakorn et al. demonstrated that more than 70\% of image-based Google reCAPTCHA and Facebook image CAPTCHA can be efficiently solved using deep learning~\cite{sivakorn2016robot,Zhou:2018:BGR:3280489.3280510}. There are studies trying to improve these schemes. For example, Wang et al. proposed to combine graphical and text-based CAPTCHAs to increase better accuracy. Walgampaya et al. designed a multi-level data fusion algorithm, which combines scores from individual clicks to generate more robust evidence, to detect click fraud~\cite{walgampaya2010real}. Nevertheless, these CAPTCHA systems require users to perform additional tasks and deliver bad user experience, especially when running on mobile devices~\cite{reynaga2013usability}. To counter this, Google reCAPTCHA v2 use a risk analysis engine to avoid interrupting users unnecessarily~\cite{Google2019}. This engine collects and analyses relevant data during click events to attest the humanness of the user. The latest reCAPTCHA v3 no longer requires users to click a button. Instead, it studies user interactions within a webpage and gives a score that represents the likelihood that a user is a human~\cite{Google2018}. Although these CAPTCHA schemes are invisible to users, a plethora of sensitive data, including cookies, browser plugins, and all JavaScript objects, is collected~\cite{LaraOReilly2015}. This data could be used to fingerprint the user browser and link user's online activities~\cite{gulyas2018extend,vastel2018fp}. To conform to data protection acts, such as the California Online Privacy Protection Act (CalOPPA) and the EU General Data Protection Regulation (GDPR), Google requires every website using reCAPTCHA to include a privacy policy to give consent to the data collection to use the service~\cite{Pegarella2018}.

With smartphones and IoT devices gaining popularity, more bot detection schemes now focus on mobile devices, where more types of embedded sensors are available. Most of these schemes requires users to perform additional motion tasks. For instance, Shrestha et al. showed that waving gestures could be used to attest the intention of users~\cite{10.1007/978-3-319-02937-5_11}. Guerar et al. designed a bot detection system that asks users to tilt their device according to the description to prove they are human~\cite{guerar2018completely}. Hupperich et al. presented a movement-based CAPTCHA scheme that requires users to perform certain gestures (e.g., hammering and fishing) using their device~\cite{10.1007/978-3-319-45572-3_3}. There are also some studies focusing on designing an invisible CAPTCHA scheme for the mobile. In particular, De Luca et al. exploited touch screen data during screen unlocking to authenticate users~\cite{DeLuca:2012:TMO:2207676.2208544}. Guerar et al. suggested a brightness-based bot prevention mechanism, BrightPass~\cite{guerar2016using}. BrightPass random generates a sequence of circles with different brightness when typing a PIN; users will input misleading lie digits in circles with low brightness. Buriro et al. proposed a behavioural-based authentication scheme for banking apps, which uses timing and device motion information during password typing to identify genuine users~\cite{buriro2017evaluation}. 

The work that is most closely related to ours is the Invisible CAPPCHA~\cite{Guerar2018}. Similar to \name, Invisible CAPPCHA leveraged the different in device acceleration between finger touch and software touch to make a decision about whether a user is a bot. It can also effectively distinguish device vibration and finger touch because they have a observably different device acceleration pattern. However, Invisible CAPPCHA only considers simple tap and vibration events; its accuracy on more complicated touch events (e.g., drag, long press, and double tap) is unclear. In comparison, \name considers more types of touch events and works regardless of the device movement. To improve the accuracy, \name uses more data source in addition to accelerometer and introduces context into the detection.

Overall, all existing CAPTCHA schemes, to the best of our knowledge, require either sending the sensor data back to the server, trusting the app code, or having a TEE on the device to execute detection logics, raising privacy, security, and usability concerns. \name is the first CAPTCHA scheme that does not make these assumptions.