\subsection{Bot Detection}

To prevent automated programs, or bots, from abusing online services, the widely adopted solution is to deploy a CAPTCHA system. The early form of CAPTCHA typically requires users to identify text from a distorted image. For Google reCAPTCHA, the most popular CAPTCHA service, user inputs are also collected to help Google digitalise printed documents~\cite{von2008recaptcha}. However, text-based CAPTCHA schemes have been proven to be insecure as machines achieved 99.8\% success rate in identifying distorted text~\cite{yan2008low,5395072,goodfellow2013multi}. Therefore, CAPTCHA service providers, such as Google, started to test image-based CAPTCHA schemes, which require users to select images that match given description~\cite{Shet2014}. In the same time, reCAPTCHA users also help Google label images for free. Nevertheless, Sivakorn et al. demonstrated that more than 70\% of image-based Google reCAPTCHA and Facebook image CAPTCHA can be efficiently solved using deep learning~\cite{sivakorn2016robot,Zhou:2018:BGR:3280489.3280510}. In addition, these CAPTCHA systems also deliver bad user experience, especially when running on mobile devices~\cite{reynaga2013usability}. To counter this, Google reCAPTCHA v2 use a risk analysis engine to avoid interrupting users unnecessarily~\cite{Google2019}. This engine collects and analyses relevant data during click events to attest the humanness of the user. The latest reCAPTCHA v3 no longer requires users to click a button. Instead, it studies user interactions within a webpage and gives a score that represents the likelihood that a user is a human~\cite{Google2018}. Although these CAPTCHA schemes are invisible to users, a plethora of sensitive data, including cookies, browser plugins, and all JavaScript objects, is collected~\cite{LaraOReilly2015}. Needless to say, these wide data collection activities raise privacy concerns. To conform to data protection acts, such as the California Online Privacy Protection Act (CalOPPA) and the EU General Data Protection Regulation (GDPR), Google requires every website using reCAPTCHA to include a privacy policy to give consent to the data collection to use the service~\cite{Pegarella2018}.