\subsection{Privacy Implications of Motion Sensor Data}

On both iOS and Android, the access to motion sensors does not require explicit user permission; the accelerometer and gyroscope can also be accessed from a mobile website via JavaScript. Previous studies have shown that this data could expose sensitive information about a user. In particular, TouchLogger~\cite{Cai:2011:TIK:2028040.2028049}, TapLogger~\cite{xu2012taplogger}, TapPrints~\cite{miluzzo2012tapprints}, and ACCessory~\cite{owusu2012accessory} can infer user inputs on a touch screen an steal user passwords based on the device acceleration data during touch events. Mehrnezhad et al. demonstrated that similar attacks can also be launched via Javascript~\cite{mehrnezhad2016touchsignatures}. In addition, extensive studies have proven that user activity can be accurately tracked from the motion data~\cite{REYESORTIZ2016754,SANSEGUNDO2018190}. Other researchers have also shown that personal user information, such as gender, age, weight, and height can be leaked from the sensory data~\cite{Malekzadeh:2018:PSD:3195258.3195260,davarci2017age}. Most recently, Zhang et al.~\cite{zhang2019sensorid} revealed that a globally unique device fingerprint can be generated from the motion sensor data. These studies strongly motivate us to design a privacy-preserving CAPTCHA scheme that does not reveal any sensitive sensor data to the server.